\chapter{Conclusions}
\label{chapter:conclusion}

\begin{introduction}
This section highlights the insights gained from the analysis of the work that is currently being developed on the topic of this dissertation, as well as how the existing solutions will support the construction of this dissertation's work
\end{introduction}


Locating archaeological sites is a difficult task, and it takes several steps to accomplish it. Over the years, methods have evolved from intensive sweeping of the terrain by people, to more sophisticated methods using lidar, for example. These modern methods can produce an enormous amount of data. This paper describes an attempt to locate archeological sites using deep learning methods. As can be seen in the paper two deep learning architectures were used, YOLOv7 and Unet. 

The whole procedure allowed us to obtain successful results, even with a small amount of initial data, and two archelogical objects with very different shapes, tumuli with a much more regular shape and size, and castros that can present various shapes and sizes.

As expected YOLOv7 worked much better for tumuli than for hillforts, since it is an object detection model. Methods like cross validation did not prove useful resulting in a huge amount of detections. it was also found that the copy paste enhancement method used worked much better with Unet than with YOLOv7, this is most likely due to Unet focusing more on the object and not the background, this being particularly useful, due to the fact that copy paste can make collages in unrealistic places.


