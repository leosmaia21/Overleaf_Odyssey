

\chapter{Conclusions and Future Work}
\label{chapter:conclusion}

\begin{introduction}
This section highlights the insights gained from the analysis of the work that is currently being developed on the topic of this dissertation, as well as how the existing solutions will support the construction of this dissertation's work
\end{introduction}

The development of a lifelog retrieval system is a complex undertaking that requires effective collaboration across various modules that address different tasks. This dissertation aims to extend a module of the MEMORIA lifelogging system by implementing different image annotation algorithms to extract relevant information from image lifelogs. Because these collected annotations will be used by other components of the system, they must be structured and optimised to provide information in a valuable way. This can be accomplished by selecting the optimal models for this particular use case.

There are numerous solutions that are currently being developed in order to advance the state of the art in various computer vision tasks that meet the aims of this work. An investigation of a part of these solutions was carried out, allowing for the exploration of potential models to be included in MEMORIA's pipeline. 

These potential models were chosen considering what type of annotations are intended to be extracted from the lifelog images. It was defined that each image should be analysed on different levels of detail, which means that lower-level annotations such as the identification of objects and optical characters should be allied to higher-level ones such as the understanding of the scene depicted in the image as well as the identification of the global meaning of the lifelog.

This completed research process serves as a starting point for the next phase of the work plan, in which each prospective model will be tested in an isolated environment to see whether it is a suitable candidate for inclusion in the retrieval system. After testing each model, they will be integrated in the system and their performance will be analysed. It is expected that the integration of these new models will result in a solution that will allow effective lifelog retrieval utilising information derived from lifelogs images submitted to the MEMORIA system.






